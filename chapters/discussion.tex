\chapter{Discussion}
In this chapter, individual case-by-case study will be discussed in detail. The results from our measurements will also be compared with the counterparts form Weder et al.

\section {Participant 3}
The results from this participant was the closest one to the results reported by Weder et al. For the oxygenated hemoglobin HbO waveform morphology, tonic response could be observed in channel 1, 2, and 3, and phasic response could be observed in channel 10 and 12.

\section {Participant 5}
For the oxygenated hemoglobin HbO waveform, there were significantly larger on-sets for the 90 dB sound stimuli in Channel 1, 2, and 3, i.e. around the Broca's area.

Apart from this, the waveforms for deoxygenated hemoglobin, HbR, were also quite different from the ones Weder et al. reported. For the loudest sound stimuli, channels overlying the caudal superior temporal gyrus and channels over Broca's area showed clear phasic response. 

\section {Participant 6}

For the oxygenated hemoglobin, HbO waveform, the loudest sound stimuli resulted in phasic response for almost all the channels. In addition, it also resulted in faster on-set compared with other stimuli of lower sound pressure levels.

On the other hand, as for the deoxygenated hemoglobin, HbR response, results from multiple channels appeared to be noisy even if the SCI values were already above the suggested threshold.


\section {Participant 7}

The results from this participant are rather indeterminant to differentiate between response to different sound pressure levels.


\section {Participant 8}
This participant was given only silence stimuli. No pattern could be concluded for the measured waveform morphology. Nonetheless, it's noteworthy to know that even if there are almost no visual and sound stimuli, dynamic hemoglobin response still exists.