\chapter{Introduction}

\section{Motivation}
This research is aimed for better understanding of the brain activities when the subjects are exposed to different audio stimuli with the help of fNIRS measurement.

In the field of neuro-imaging, although fMRI is widely used and provides excellent (spatial) resolution, it still has many limitations, especially when it comes to hearing research. First of all, MRI rooms are noisy, which makes it difficult to control the audio stimulation desired due to inevitable environmental noises. In addition, fMRI scans are done in a magnetic field. It has not yet been proved that pregnant women and infants can be safely exposed to an external magnetic field in the MRI room. For people with hearing disabilities, more specifically cochlear implant patients, going into a MRI room would not be ideal, either. Although there are already cochlear implants that can be worn to a magnetic field, it is still generally not suggested to wear a piece of metal in a MRI room.

fNIRS,  short for functional near-infrared spectroscopy. With fNIRS, we can measure brain activity by using near-infrared light to estimate cortical hemodynamic activity which occur in response to neural activity. It is non-invasive and risk-free. The fNIRS device is portable and works silently. With the cap secured on the head, it is also more resilient to motion artifacts. All these makes it ideal for hearing researches. However, it is not yet commonly used in clinical diagnostics due to the lack of understanding of the expected brain activities measured with fNIRS. Therefore, in this research, we'd like to perform some fNIRS measurement and analyse the fNRIS data under different experiment conditions.

If fNIRS can provide more meaningful data and be more commonly used in early clinical diagnosis, we may find hearing abnormality of patients earlier. This is especially important for infants or children. As language development happens in the early stages of one's life, the sooner we find the hearing abnormality and fix it, the better. After a child turns 8, it is practically not possible for him to understand human speech even with perfect hearing. I personally find hearing research a meaningful topic. For one, speech is the primary and direct way of human communication. We express ourselves and perceive other people's opinion via speech. For the other, music has always been an important part of my life for me personally. Without the ability to hear and listen, neither speech nor music will be possible to be perceived. Therefore, I want to help other people with hearing disabilities get better diagnosis and treatment. fNIRS is of great potential to help solve the issue.

\section{Technical Background}
Hemoglobin, the protein from inside red blood cells, transports oxygen molecules throughout the body. Higher hemoglobin levels and red blood cell transfusion are associated with higher cerebral oxygen delivery. Different concentration levels of hemoglobin results in a spectral change. The biological tissue has a relatively good transparency for light in the near-infrared region (700-1300nm) \cite{doi:10.1126/science.929199}. Therefore, it's possible to transmit sufficient photons in situ monitoring. 

The tenichque of NIRS relies on the Beer-Lambert law, which is givien by:
\[
OD_{\lambda} = Log(\frac {I_0}{I}) = \epsilon _{\lambda} \cdot c \cdot L
\]
$OD_{\lambda} $: a dimensionless factor know as the optical density of the medium.  \\
$I_0$ : the incident radiation. \\
$I$: the transmitted radiaion. \\
$\epsilon _{\lambda}$: the molar absorptivity ($mM^{-1} \cdot cm^{-1}$) of the chromophore. \\
$c$: the concentration ($mM$)of the chromophore. \\
$L$: length of light path. \\

The Beer-Lambert law was intended for use in a clear, non-scattering medium. When the law is applied to a scattering medium, e.g. brain tissue, a correction factor should be applied. The factor, called "differential pathlength factor (DPF)" accounts for the increase in optical path length due to scattering in the tissue. The modified Beer-Lambert law is given by:
\[
OD_{\lambda} = \epsilon _{\lambda} \cdot c \cdot L \cdot B + OD_{R,L}
\]
where $OD_{R,L}$ represents the oxygen-independent light absorption due to scattering in the tissue, and B is the mean pathlength traveled by the detected photons. In our case, i.e. CW-NIRS, this mean pathlength is not knowm. In a highly scattering medium, the pathlength of trajectories is longer than the source-detector seperation. Nevertheless, one may still estimate the pathlength within the whole sampling region by multiplying the source-detector distance with a DPF. Assuming $OD_{R,L}$ is constant during a measurement, we may rewrite the previous equation in terms of changes in optical density and changes in concentration as follows:
\[
\Delta c =\frac { \Delta OD_{\lambda}} {\epsilon _{\lambda} \cdot L \cdot B}
\]
The validity of the above equation depends on how much B varies. \cite {Delpy_1988} investigated this question and gave a relation between the DPF and the head diameter. Nonetheless, newer research also provides different ways to estimate the DPF. In the scope of this project, the DPF was calculated from a function of wavelengths and age of the participant \cite {Duncan1996MeasurementOC}.







%%% Local Variables:
%%% mode: latex
%%% TeX-master: "../thesis"
%%% End:
