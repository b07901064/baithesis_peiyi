\chapter{Results}

From our measurements, the results varied a lot individually. Hence, grand average and further statistical analysis would not be well-applicable. In this section, individual results from selected participants are presented, including waveform morphology of the 14-channel measurements for both the HbO and HbR data. In addition, the regional analysis for the HbO data are also included in this section. Results from other measured participants are included in appendix.

First of all, our channels with the optode template are defined as shown in figure ~\ref{fig:ChannelDef}.

\begin{figure}[H]
  \centering
    \includegraphics[scale=.45]{bilder/optode_ink.png}
  \caption{Channel Definition}
  \label{fig:ChannelDef}
\end{figure}


Our regions of interest (\acrshort{roi}) were defined as the following figure. The auditory cortex was in particular of our interest. Hence, channel 4, channel 8, and channel 9 together formed one region (ROI 2). The rest of the channels formed ROI 1. It was of our interest to compare how the response of the auditory cortex differ from the rest of the left brain hemisphere.


\vspace{1cm}
\begin{figure}[H]
  \centering
    \includegraphics[scale=.45]{bilder/optode_roi_ink.png}
  \caption{ROI Definition}
\end{figure}


In the following plots. Channels with invalid SCI would not be taken into consideration, and hence would not be shown. Measurements in all channels were plotted in the same scale except for the two short channels marked in thicker outlines. In all our measurements, the changes in the dynamic hemoglobin response were significantly less in the short channels by more than a magnitude.
\newpage



\section {Participant 3}

\begin{figure}[H]
  \centering
    \includegraphics[scale=.4]{bilder/HbO_Mole/sub_jonas_s_HbO.png}
  \caption{HbO Measurement from participant 3.}
  \label{fig:somesignal}
\end{figure}

\begin{figure}[H]
  \centering
    \includegraphics[scale=.4]{bilder/HbR_Mole/sub_jonas_s_HbR.png}
  \caption{HbR Measurement from participant 3.}
  \label{fig:somesignal}
\end{figure}

\begin{figure}[H]
  \centering
    \includegraphics[scale=.3]{bilder/ROI/sub_jonas_s_HbO.png}
  \caption{ROI Measurement from participant  3.}
\end{figure}

The results from this participant was the closest one to the results reported by Weder et al. For the oxygenated hemoglobin HbO waveform morphology, tonic response could be observed in channel 1, 2, and 3, and phasic response could be observed in channel 10 and 12.

\newpage


\section {Participant 4}
There were also some poor measurements even though the SCI is above the threshold 0.75. For example, in our case of participant 4. One possible reason can be due to the thick dark hair of the participant. Light absorption can affect the result greatly.

\begin{figure}[H]
  \centering
    \includegraphics[scale=.35]{bilder/HbO_Mole/sub_lin_s_HbO.png}
  \caption{HbO Measurement from participant 4.}
\end{figure}


\begin{figure}[H]
  \centering
    \includegraphics[scale=.35]{bilder/HbR_Mole/sub_lin_s_HbR.png}
  \caption{HbR Measurement from participant 4.}
\end{figure}

\begin{figure}[H]
  \centering
    \includegraphics[scale=.29]{bilder/ROI/sub_lin_s_HbO.png}
  \caption{ROI Measurement from participant  4.}
\end{figure}

\newpage



\section {Participant 5}
\begin{figure}[H]
  \centering
    \includegraphics[scale=.4]{bilder/HbO_Mole/sub_lukas_s_HbO.png}
  \caption{HbO Measurement from participant 5.}
  \label{fig:somesignal}
\end{figure}

\begin{figure}[H]
  \centering
    \includegraphics[scale=.4]{bilder/HbR_Mole/sub_lukas_s_HbR.png}
  \caption{HbR Measurement from participant 5.}
  \label{fig:somesignal}
\end{figure}

\begin{figure}[H]
  \centering
    \includegraphics[scale=.29]{bilder/ROI/sub_lukas_s_HbO.png}
  \caption{ROI Measurement from participant 5.}
\end{figure}

For the \acrlong{HbO} (\acrshort{HbO} ) waveform, there were significantly larger on-sets for the 90 dB sound stimuli in Channel 1, 2, and 3, i.e. around the Broca's area.

Apart from this, the waveforms for \acrlong{HbR} (\acrshort{HbR}), were also quite different from the ones Weder et al \citeyearpar{Weder2018}. reported. For the loudest sound stimuli, channels overlying the caudal superior temporal gyrus and channels over Broca's area showed clear phasic response. 


\newpage




\section {Participant 6}
\begin{figure}[H]
  \centering
    \includegraphics[scale=.4]{bilder/HbO_Mole/sub_shelia_s_HbO.png}
  \caption{HbO Measurement from participant 6.}
  \label{fig:somesignal}
\end{figure}

\begin{figure}[H]
  \centering
    \includegraphics[scale=.4]{bilder/HbR_Mole/sub_shelia_s_HbR.png}
  \caption{HbR Measurement from participant 6.}
  \label{fig:somesignal}
\end{figure}

\begin{figure}[H]
  \centering
    \includegraphics[scale=.29]{bilder/ROI/sub_shelia_s_HbO.png}
  \caption{ROI Measurement from participant  6.}
\end{figure}

For the oxygenated hemoglobin, \acrshort{HbO} waveform, the loudest sound stimuli resulted in phasic response for almost all the channels. In addition, it also resulted in faster on-set compared with other stimuli of lower sound pressure levels.

On the other hand, as for the deoxygenated hemoglobin, \acrshort{HbR} response, results from multiple channels appeared to be noisy even if the SCI values were already above the suggested threshold.

\newpage





\section {Participant 7}
\begin{figure}[H]
  \centering
    \includegraphics[scale=.4]{bilder/HbO_Mole/sub_liao_s_HbO.png}
  \caption{HbO Measurement from participant 7.}
  \label{fig:somesignal}
\end{figure}

\begin{figure}[H]
  \centering
    \includegraphics[scale=.4]{bilder/HbR_Mole/sub_liao_s_HbR.png}
  \caption{HbR Measurement from participant 7.}
  \label{fig:somesignal}
\end{figure}

\begin{figure}[H]
  \centering
    \includegraphics[scale=.29]{bilder/ROI/sub_liao_s_HbO.png}
  \caption{ROI Measurement from participant 7.}
\end{figure}

The results from this participant are rather indeterminant to differentiate between response to different sound pressure levels.

\newpage








%The following plots shows the averaged \textbf {HbO} response of all the valid channels in the defined region.

