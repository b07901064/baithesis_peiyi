\chapter{Conclusion and Future Prospectives}
%//conclusion
This study aims to confirm and reproduce the results from Weder et al. \citeyear{Weder2018}. With the limited devices, we made our measurement conditions, i.e. optode template configuration and audio sound stimuli as similar as those of the previous research from Weder et al. \citeyear{Weder2018} as possible.

Our results were not completely as expected. Few common patterns could be found between participants. Therefore,  we analysed the results from each individual participant and compared these with the results Weder et al. \citeyear{Weder2018} provided. Variables that might have potentially affected the results of the study were investigated with a thorough literature review.

Other than purely reproducing the results from the one main paper \cite{Weder2018}, this study also take other related research into consideration.  We studied and considered the effect of different audio stimuli, individual perception of loudness of the sound, depth, laterality and location of the cortical activation. By taking advantage of previous fMRI studies \cite{Belin2000} \cite{Belin2002} \cite{Hall} \cite{Frost1999-vs}, we compared the laterality of cortical activation in this study and previous researches. Building upon the presumption from earlier researches and our measurement data, we also concluded that the cortical activation from audio stimuli is possibly deeper for some participants in the audio cortex. Additionally other than merely taking the HbO data for analysis, the magnitude and noise of the measured HbR data was also compared with the counterparts of the HbO data.