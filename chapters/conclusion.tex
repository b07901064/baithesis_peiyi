\chapter{Conclusion and Future Prospectives}
%//conclusion
In this project, we aimed to confirm and reproduce the results from Weder et al. With the limited devices, we made our measurement conditions, i.e. optode template configuration and audio sound stimuli as similar as those of the previous research from Weder et al. as possible.

The results we got for the waveform morphologies did not speak entirely with the results that Weder et al. reported. In most of the cases, larger sound pressure level did result in greater positive change of oxygenated hemoglobin concentration, or in other words, greater negative change in deoxygenated hemoglobin concentration. However, the results were not consistent between participants. The separation between different sound pressure levels could not be clearly seen. Apart from the results with the loudest audio stimuli, responses from other quieter audio stimuli were rather indistinguishable. Moreover, regarding the type of responses we measured from different regions of the left brain hemisphere, phasic response could be observed from the channels over the supramarginal superior temporal gyrus from most of the participants. However, only from some of the paritcipants, channels over Broca's area could show a broad tonic pattern as Weder et al. described.

There could be several factors that potentially caused the results from this project to vary from that Weder et al. reported. First, we used the device Brite23. It is also a continuous-wave fNIRS device. The sources emitted light of slightly different wavelengths, which are 757 nm and 843 nm, whereas Weder et al. uses the device (NIRScout, NIRX, Germany) which sources emitted light of wavelengths 760 nm and 850 nm. Additionally as for the fNIRS testing procedure, we could have also improved on several things. To begin with, a darken sound booth would be more suitable. In our setup, the testing was also performed in a sound booth, but with normal light condition. Still, if the lighting was dimmer, there could be less noise in the hemoglobin response from visual stimulation. In addition, it would make sense to stabilize the participant's neck with a neck cushion. Not only would it be more comfortable for the participants during the measurement. Movement artifacts could also be reduced. Next, the measured data was processed with a modified approach. In the research from Weder et al., data pre-processing and analysis was executed in MATLAB and SPSS (version 24, IBM Corp., USA). They combined custom-made MATLAB scripts with Homer2 functions. On the other hand, we chose to use the newer Homer3 with our MATLAB script. Judging from the varying individual results, group analysis or statistical analysis will not be applicable in our case. Hence, the software program for statistical analysis, SPSS, was not used in this project. Furthermore, we calculated the differential pathlength factor (DPF) for each participant from their age, and calculate the resultant HbO and HbR concentration with the correction factor, whereas Weder et al. did not mention how they chose or calculated the DPF values.

For the regional analysis, we did not group the regions of interests like how Weder et al. did. We did not group them depending on the measured waveform, either. In fact, response from the auditory cortex was more of our interest. As a result, we separated the three channels over caudal superior temporal gyrus apart from the rest of the channels and compared their averaged hemoglobin responses. The results showed that from our measurements,  the response from the auditory cortex speak very close with that from the rest parts of the left brain hemisphere. 

Speaking from the experience gained from this project, there are several things that can be improved for future research. Other than the above-mentioned fNIRS testing procedures with darker sound booth and additional neck cushion. The way how the cap was put on and adjusted can still be improved. Having more reference points will surly be helpful and make it easier to position the cap more accurately. In my opinion, we could have had a better understanding of the spatial position with the optode templates better before we started measuring people. That is in other words, not just positioning the cap according to the anchor points, but also knowing which channels covers which regions of the brain at the same time. Also, from our measurements, data from female participants with long hair had worse data quality. In our configuration, since we were only measuring the left brain hemisphere. First asking the participant to put the hair to the right side made it easier to put the cap on. Sometimes when the participants had thick long hair, trying to put the hair aside can be a futile attempt, but it was easier when they first put the hair to the other side. Last but not least, we would be curious to know how would the hemoglobin response from more people be like. If possible, more participants should be measured so the results from the project can be more credible. For example, participants of different ages and every gender and race would be desirable for this hearing research. Besides, other than only measure normal-hearing people, it would also be of our great interest to measure some cochlear implant users and compare the results together.