\chapter{Conclusion and Future Prospectives}
%//conclusion
In this project, we aimed to confirm and reproduce the results from Weder et al. With the limited devices, we made our measurement conditions, i.e. optode template configuration and audio sound stimuli as similar as those of the previous research from Weder et al. as possible.

The results we got for the waveform morphologies did not speak entirely with the results that Weder et al. reported. In most of the cases, larger sound pressure level did result in greater positive change of oxygenated hemoglobin concentration, or in other words, greater negative change in deoxygenated hemoglobin concentration. However, the results are not consistent between every participants. The separation between different sound pressure levels could not be clearly seen. Apart from the results with the loudest audio stimuli, responses from other quieter audio stimuli were rather indistinguishable. Moreover, regarding the type of responses we measured from different regions of the left brain hemisphere, phasic response could be observed from the channels over the supramarginal superior temporal gyrus from most of the participants. However, only from some of the paritcipants, channels over Broca's area could show a broad tonic pattern as Weder et al. described.

There could be several factors which caused the results from this project to vary from that Weder et al. reported. First, we used the device Brite23. It is also a continuous-wave fNIRS device. The sources emitted light of slightly different wavelengths, which are 757 nm and 843 nm, whereas Weder et al. uses the device (NIRScout, NIRX, Germany) which sources emitted light of wavelengths 760 nm and 850 nm. Additionally as for the fNIRS testing procedure, we could have also improved on several things. To begin with, a darken sound booth would be more suitable. In our setup, the testing was also performed in a sound booth, but with normal light condition. However, if the lighting was dimmer, there could be less noise in the hemoglobin response from visual stimulation. In addition, it would make sense to stabilize the participant's neck with a neck cushion. Not only would it be more comfortable for the participants during the measurement. Movement artifacts could also be reduced. Next, the measured data was processed with modified approach.

data processing
-> almost
->still not yet mentioned the beta and short channel extra cebrelle thingy
->dpf

waveform 

roi

//future

what could have gone wrong:
software different
cap position (optode template design and offsets when putting on a cap) mention the broca's area having phasic response (jonas?)

different device, different wavelength

more participants

better understanding of the spatial position with the optode templates (e.g. which channels are over brocas area, which are over supramarginal or caudal superior temporal gyrus, etc.