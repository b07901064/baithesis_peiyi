\chapter{Conclusion and Future Prospectives}
%//conclusion
This study aims to confirm and reproduce the results from Weder et al. \citeyearpar {Weder2018}. With the limited devices, we made our measurement conditions, i.e. optode template configuration and audio sound stimuli as similar as those of the previous research from Weder et al. \citeyearpar{Weder2018} as possible.

Our results were not completely as expected. Few common patterns could be found between participants. Therefore,  we analysed the results from each individual participant and compared these with the results Weder et al. \citeyearpar{Weder2018} provided. Variables that might have potentially affected the results of the study were investigated with a thorough literature review.

Other than purely attempting to reproduce the results from the one main paper \citep{Weder2018}, this study also takes other related research into consideration.  We studied and considered the effect of different auditory stimuli, individual perception of the loudness of the sound, depth, laterality, and location of the cortical activation. By taking advantage of previous fMRI studies \citep{Belin2000} \citep{Belin2002} \citep{Hall} \citep{Frost1999-vs}, we compared the laterality of cortical activation in this study and previous research. Building upon the presumption from earlier research and our measurement data, we also concluded that the cortical activation from auditory stimuli is possibly deeper for some participants in the auditory cortex. Additionally other than merely taking the HbO data for analysis, the magnitude and noise of the measured HbR data were also compared with the counterparts of the HbO data.

In this study, only normal-hearing adults were enrolled in order to best dine the capabilities of fNIRS in this straightforward situation. Nevertheless, a long-term goal of this research is to use fNIRS to objectively evaluate how auditory stimuli relay to the auditory cortex in deaf subjects before and after cochlear implantation. The vision is to be able to use fNIRS in clinical applications to identify and intervene in hearing loss earlier in child development. Hence, we hope to include more subjects which not only include normal-hearing adults but also cochlear implant users and children. A comparison between normal-hearing adults and other groups can provide value in this field of study. Additionally, it would also be of our interest to investigate the cortical response with fNIRS when subjects are given other different stimuli, e.g. visual-only stimuli, normal-speech stimuli, or musical stimuli.